\hypertarget{_blackman_8h_source}{\section{Blackman.\+h}
\label{_blackman_8h_source}\index{Blackman.\+h@{Blackman.\+h}}
}

\begin{DoxyCode}
00001 \textcolor{comment}{//}
00002 \textcolor{comment}{//  Blackman.h}
00003 \textcolor{comment}{//  DSG}
00004 \textcolor{comment}{//}
00005 \textcolor{comment}{//  Created by Alexander Zywicki on 9/24/14.}
00006 \textcolor{comment}{//  Copyright (c) 2014 Alexander Zywicki. All rights reserved.}
00007 \textcolor{comment}{//}
00008 \textcolor{comment}{/*}
00009 \textcolor{comment}{ This file is part of the Digital Signal Generation Project or “DSG”.}
00010 \textcolor{comment}{}
00011 \textcolor{comment}{ DSG is free software: you can redistribute it and/or modify}
00012 \textcolor{comment}{ it under the terms of the GNU General Public License as published by}
00013 \textcolor{comment}{ the Free Software Foundation, either version 3 of the License, or}
00014 \textcolor{comment}{ (at your option) any later version.}
00015 \textcolor{comment}{}
00016 \textcolor{comment}{ DSG is distributed in the hope that it will be useful,}
00017 \textcolor{comment}{ but WITHOUT ANY WARRANTY; without even the implied warranty of}
00018 \textcolor{comment}{ MERCHANTABILITY or FITNESS FOR A PARTICULAR PURPOSE.  See the}
00019 \textcolor{comment}{ GNU General Public License for more details.}
00020 \textcolor{comment}{}
00021 \textcolor{comment}{ You should have received a copy of the GNU General Public License}
00022 \textcolor{comment}{ along with DSG.  If not, see <http://www.gnu.org/licenses/>.}
00023 \textcolor{comment}{ */}
00024 \textcolor{preprocessor}{#ifndef DSG\_Blackman\_h}
00025 \textcolor{preprocessor}{#define DSG\_Blackman\_h}
00026 \textcolor{preprocessor}{#include "\hyperlink{_p_i_8h}{PI.h}"}
00027 \textcolor{preprocessor}{#include "\hyperlink{_l_u_t_8h}{LUT.h}"}
00028 \textcolor{preprocessor}{#include "\hyperlink{_sine_8h}{Sine.h}"}
00029 \textcolor{keyword}{namespace }\hyperlink{namespace_d_s_g}{DSG} \{
00030 \textcolor{preprocessor}{#ifdef DSG\_Short\_Names}
00031     \textcolor{keyword}{inline}
00032 \textcolor{preprocessor}{#endif}
\hypertarget{_blackman_8h_source_l00033}{}\hyperlink{namespace_d_s_g_1_1_window}{00033}     \textcolor{keyword}{namespace }Window\{\textcolor{comment}{}
00034 \textcolor{comment}{        //!\(\backslash\)brief DSG::Window::Blackman - Blackman Window Function}
00035 \textcolor{comment}{}        \textcolor{keyword}{template}<\textcolor{keyword}{typename} decimal>
\hypertarget{_blackman_8h_source_l00036}{}\hyperlink{namespace_d_s_g_1_1_window_a0800636ec7008aa75ff987feef5aafdf}{00036}         \textcolor{keyword}{inline} decimal \hyperlink{namespace_d_s_g_1_1_window_a0800636ec7008aa75ff987feef5aafdf}{Blackman}(decimal \textcolor{keyword}{const}& x)\{
00037             \textcolor{comment}{// Generate Blackman Window}
00038             \textcolor{comment}{/*}
00039 \textcolor{comment}{             Blackman(x) = 0.42f - (0.5f * cos(2pi*x)) + (0.08f * cos(2pi*2.0*x));}
00040 \textcolor{comment}{             \}*/}
00041             static\_assert(std::is\_floating\_point<decimal>::value==\textcolor{keyword}{true},\textcolor{stringliteral}{"DSG::Blackman Function Requires
       Floating Point Type"});
00042             \textcolor{comment}{//we will implement the blackman window as a function as if it were sin(x)}
00043             \textcolor{comment}{//cos input domain 0-1 not 0-2pi}
00044             \textcolor{comment}{//range checking is handles within DSG::Cos}
00045             decimal phs=x;
00046             \textcolor{keywordflow}{while} (phs>1.0) \{
00047                 phs-=1.0;
00048             \}
00049             \textcolor{keywordflow}{return} 0.42 - (0.5 * \hyperlink{namespace_d_s_g_ade303ad15c77f534429305c3cbd90191}{DSG::Cos}(phs))+(0.08 * \hyperlink{namespace_d_s_g_ade303ad15c77f534429305c3cbd90191}{DSG::Cos}(2.0*phs));
00050         \}
00051     \}
00052 \}
00053 \textcolor{preprocessor}{#endif}
\end{DoxyCode}
