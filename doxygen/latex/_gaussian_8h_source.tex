\hypertarget{_gaussian_8h_source}{\section{Gaussian.\+h}
\label{_gaussian_8h_source}\index{Gaussian.\+h@{Gaussian.\+h}}
}

\begin{DoxyCode}
00001 \textcolor{comment}{//}
00002 \textcolor{comment}{//  Gaussian.h}
00003 \textcolor{comment}{//  DSG}
00004 \textcolor{comment}{//}
00005 \textcolor{comment}{//  Created by Alexander Zywicki on 10/6/14.}
00006 \textcolor{comment}{//  Copyright (c) 2014 Alexander Zywicki. All rights reserved.}
00007 \textcolor{comment}{//}
00008 \textcolor{comment}{/*}
00009 \textcolor{comment}{ This file is part of the Digital Signal Generation Project or “DSG”.}
00010 \textcolor{comment}{}
00011 \textcolor{comment}{ DSG is free software: you can redistribute it and/or modify}
00012 \textcolor{comment}{ it under the terms of the GNU General Public License as published by}
00013 \textcolor{comment}{ the Free Software Foundation, either version 3 of the License, or}
00014 \textcolor{comment}{ (at your option) any later version.}
00015 \textcolor{comment}{}
00016 \textcolor{comment}{ DSG is distributed in the hope that it will be useful,}
00017 \textcolor{comment}{ but WITHOUT ANY WARRANTY; without even the implied warranty of}
00018 \textcolor{comment}{ MERCHANTABILITY or FITNESS FOR A PARTICULAR PURPOSE.  See the}
00019 \textcolor{comment}{ GNU General Public License for more details.}
00020 \textcolor{comment}{}
00021 \textcolor{comment}{ You should have received a copy of the GNU General Public License}
00022 \textcolor{comment}{ along with DSG.  If not, see <http://www.gnu.org/licenses/>.}
00023 \textcolor{comment}{ */}
00024 \textcolor{preprocessor}{#ifndef DSG\_Gaussian\_h}
00025 \textcolor{preprocessor}{#define DSG\_Gaussian\_h}
00026 \textcolor{preprocessor}{#include "\hyperlink{_sine_8h}{Sine.h}"}
00027 \textcolor{preprocessor}{#include "\hyperlink{_white_8h}{White.h}"}
00028 \textcolor{keyword}{namespace }\hyperlink{namespace_d_s_g}{DSG}\{
00029 \textcolor{preprocessor}{#ifdef DSG\_Short\_Names}
00030     \textcolor{keyword}{inline}
00031 \textcolor{preprocessor}{#endif}
\hypertarget{_gaussian_8h_source_l00032}{}\hyperlink{namespace_d_s_g_1_1_noise}{00032}     \textcolor{keyword}{namespace }Noise\{\textcolor{comment}{}
00033 \textcolor{comment}{        //!\(\backslash\)brief DSG::Noise::Gaussian - Gaussian Noise Generator Function}
00034 \textcolor{comment}{}        \textcolor{keyword}{template}<\textcolor{keyword}{typename} decimal=DSG::DSGSample>
\hypertarget{_gaussian_8h_source_l00035}{}\hyperlink{namespace_d_s_g_1_1_noise_a87c4bcd92a902d32df1d7f1d5acffcd4}{00035}         decimal \hyperlink{namespace_d_s_g_1_1_noise_a87c4bcd92a902d32df1d7f1d5acffcd4}{Gaussian}(decimal=0.0)\{
00036             \textcolor{keyword}{static} decimal normalizer=1;\textcolor{comment}{//variable used to actively normalize the output}
00037             \textcolor{comment}{//to enforce compatability with DSG::LUT a dummy parameter is applied}
00038             \textcolor{comment}{//this parameter is useless except for compatability reasons}
00039             decimal R1 = \hyperlink{namespace_d_s_g_1_1_noise_a0d1c4b4522d2e56b1aa604e45ab92066}{DSG::Noise::White}();
00040             decimal R2 = \hyperlink{namespace_d_s_g_1_1_noise_a0d1c4b4522d2e56b1aa604e45ab92066}{DSG::Noise::White}();
00041             decimal x= (decimal)sqrt(-2.0f * log(R1))*\hyperlink{namespace_d_s_g_ade303ad15c77f534429305c3cbd90191}{DSG::Cos}(R2);
00042             \textcolor{keywordflow}{if} (\hyperlink{namespace_d_s_g_a0af03bade7e25e8da80e3022af0e45a7}{DSG::Abs}(x)>normalizer) \{
00043                 \textcolor{comment}{//store highest output}
00044                 normalizer=\hyperlink{namespace_d_s_g_a0af03bade7e25e8da80e3022af0e45a7}{DSG::Abs}(x);
00045             \}
00046             x/=normalizer;\textcolor{comment}{//normalize}
00047             \textcolor{keywordflow}{return} x;
00048         \}
00049     \}
00050 \}
00051 \textcolor{preprocessor}{#endif}
\end{DoxyCode}
